{\bfseries g\+Sparse} is a C++ library for Graph Sparsification. It is built on top of \href{http://eigen.tuxfamily.org}{\tt Eigen}, an open source linear algebra library.

\subsection*{Sparsification Algorithm(s)}

{\bfseries g\+Sparse} currently supports the following algorithm(s).
\begin{DoxyItemize}
\item \href{https://arxiv.org/abs/0803.0929}{\tt Spectral Sparsification by Effective Resistance}
\end{DoxyItemize}

\subsection*{Header-\/only Library}

{\bfseries g\+Sparse} is implemented as a header-\/only C++ library, whose only dependency, {\bfseries Eigen}, is also header-\/only.

\subsection*{License}

{\bfseries g\+Sparse} is an open source project licensed under M\+IT License.

\subsection*{Examples-\/1}

An example that demonstrates the g\+Sparse to sparsify a graph from C\+SV files


\begin{DoxyCode}
\textcolor{preprocessor}{#include <gSparse/gSparse.hpp>}
\textcolor{preprocessor}{#include <iostream>}

\textcolor{keywordtype}{int} main()
\{
    \textcolor{comment}{// Create a CSV Reader and specify the file location}
    gSparse::GraphReader csvReader = std::make\_shared<gSparse::GraphCSVReader>(\textcolor{stringliteral}{"edgeList.csv"}, \textcolor{stringliteral}{"
      weightList.csv"},\textcolor{stringliteral}{","});
    \textcolor{comment}{// Create an Undirected Graph. gSparse::Graph object is a shared\_ptr.}
    gSparse::Graph graph = std::make\_shared<gSparse::UndirectedGraph>(csvReader);

    \textcolor{comment}{// Display original edge count}
    std::cout<<\textcolor{stringliteral}{"Original Edge Count: "} << graph->GetEdgeCount() << std::endl;

    \textcolor{comment}{// Creating Sparsifier Object}
    \mbox{\hyperlink{classg_sparse_1_1_spectral_sparsifier_1_1_e_r_sampling}{gSparse::SpectralSparsifier::ERSampling}} sparsifier(graph);
    \textcolor{comment}{// Set Hyper-parameters}
    \textcolor{comment}{// Approximate the Effective Weight Resistance (Faster)}
    \textcolor{comment}{// C = 4 and Epsilon = 0.5}
    sparsifier.SetERPolicy(gSparse::SpectralSparsifier::APPROXIMATE\_ER);
    sparsifier.SetC(4.0);
    sparsifier.SetEpsilon(0.5);
    \textcolor{comment}{// Compute Effective Weight Resistance}
    sparsifier.Compute();
    \textcolor{comment}{// Perform Sparsification }
    \textcolor{keyword}{auto} sparseGraph1 = sparsifier.GetSparsifiedGraph();
    std::cout<<\textcolor{stringliteral}{"Sparised Edge Count (ApproxER): "} << sparseGraph1->GetEdgeCount() << std::endl;
    \textcolor{keywordflow}{return} EXIT\_SUCCESS;
\}
\end{DoxyCode}


\subsection*{Examples-\/2}

An example that shows how to create a Graph object from C\+SV file and Eigen Matrix


\begin{DoxyCode}
\textcolor{preprocessor}{#include <gSparse/gSparse.hpp>}
\textcolor{preprocessor}{#include <iostream>}
\textcolor{preprocessor}{#include <memory>}

\textcolor{keywordtype}{int} main()
\{
    \textcolor{comment}{// Create a CSV Reader and specify the file location}
    gSparse::GraphReader csvReader = std::make\_shared<gSparse::GraphCSVReader>(\textcolor{stringliteral}{"
      Examples/Example-1-edges.csv"}, \textcolor{stringliteral}{"Examples/Example-1-weights.csv"});
    \textcolor{comment}{// Create an Undirected Graph. gSparse::Graph object is a shared\_ptr.}
    gSparse::Graph graph = std::make\_shared<gSparse::UndirectedGraph>(csvReader);

    \textcolor{comment}{// Display the Edge list of the graph}
    std::cout<<\textcolor{stringliteral}{"Edge List of Graph"}<<std::endl;
    std::cout<<graph->GetEdgeList()<<std::endl;
    \textcolor{comment}{// Display the weight list of the graph}
    std::cout<<\textcolor{stringliteral}{"Weight List of Graph"}<<std::endl;
    std::cout<<graph->GetWeightList()<<std::endl;

    \textcolor{comment}{// Creating Edge and Weight Data Manually using Eigen Matrix}
    gSparse::EdgeMatrix Edges(3, 2);
    gSparse::PrecisionMatrix Weights(3, 1);
    Edges(0, 0) = 0; Edges(0, 1) = 1;
    Edges(1, 0) = 1; Edges(1, 1) = 2;
    Edges(2, 0) = 2; Edges(2, 1) = 3;
    Weights << 1, 1, 1;

    \textcolor{comment}{// Construct the graph based on the data generated}
    \mbox{\hyperlink{classg_sparse_1_1_undirected_graph}{gSparse::UndirectedGraph}} test(Edges);
    \mbox{\hyperlink{classg_sparse_1_1_undirected_graph}{gSparse::UndirectedGraph}} validation(Edges,Weights);

    \textcolor{keywordflow}{return} EXIT\_SUCCESS;
\}
\end{DoxyCode}


\subsection*{Examples-\/3}

An example that shows how to calculate graph effective resistance.


\begin{DoxyCode}
\textcolor{preprocessor}{#include <gSparse/gSparse.hpp>}
\textcolor{preprocessor}{#include <iostream>}

\textcolor{keywordtype}{int} main()
\{

    \textcolor{comment}{// Create a Unit Complete Graph}
    \textcolor{keyword}{auto} graph = gSparse::Builder::buildUnitCompleteGraph(5);

    \textcolor{comment}{// Creating object to calculate Exact ER (slow)}
    \mbox{\hyperlink{classg_sparse_1_1_e_r_1_1___exact_e_r}{gSparse::ER::ExactER}} exactER;
    \textcolor{comment}{// Object for calculate Approximate ER (faster)}
    \mbox{\hyperlink{classg_sparse_1_1_e_r_1_1___approximate_e_r}{gSparse::ER::ApproximateER}} approxER; 

    \textcolor{comment}{// Object to hold EffectiveResistance data}
    gSparse::PrecisionRowMatrix er;

    \textcolor{comment}{// Calculate Exact ER}
    exactER.\mbox{\hyperlink{classg_sparse_1_1_e_r_1_1___exact_e_r_a47b950a81c815626a9d51a6284d1d49d}{CalculateER}}(er, graph );
    std::cout<<\textcolor{stringliteral}{"Exact ER Results"}<<std::endl;
    std::cout<<\textcolor{stringliteral}{"----------------"}<<std::endl;
    std::cout<<er<<std::endl;

    \textcolor{comment}{// Calculate Approximate ER}
    approxER.\mbox{\hyperlink{classg_sparse_1_1_e_r_1_1___approximate_e_r_abf16cea687d1129e1a13b4af0db44892}{CalculateER}}(er, graph);
    std::cout<<\textcolor{stringliteral}{"Approx ER Results"}<<std::endl;
    std::cout<<\textcolor{stringliteral}{"----------------"}<<std::endl;
    std::cout<<er<<std::endl;
    \textcolor{keywordflow}{return} EXIT\_SUCCESS;
\}
\end{DoxyCode}


\subsection*{Examples-\/4}

This example shows you how to perform Spectral Sparsification by Effective Weight Resistance over completed graph set.


\begin{DoxyCode}
\textcolor{preprocessor}{#include <gSparse/gSparse.hpp>}
\textcolor{preprocessor}{#include <iostream>}

\textcolor{keywordtype}{int} main()
\{
    \textcolor{comment}{// Generate 100x100 Complete Graph}
    \textcolor{keyword}{auto} graph = gSparse::Builder::buildUnitCompleteGraph(100);

    \textcolor{comment}{// Display original edge count}
    std::cout<<\textcolor{stringliteral}{"Original Edge Count: "} << graph->GetEdgeCount() << std::endl;

    \textcolor{comment}{// Creating Sparsifier Object}
    \mbox{\hyperlink{classg_sparse_1_1_spectral_sparsifier_1_1_e_r_sampling}{gSparse::SpectralSparsifier::ERSampling}} sparsifier(graph);
    \textcolor{comment}{// Set Hyper-parameters}
    \textcolor{comment}{// Approximate the Effective Weight Resistance (Faster)}
    \textcolor{comment}{// C = 4 and Epsilon = 0.5}
    sparsifier.SetERPolicy(gSparse::SpectralSparsifier::APPROXIMATE\_ER);
    sparsifier.SetC(4.0);
    sparsifier.SetEpsilon(0.5);
    \textcolor{comment}{// Compute Effective Weight Resistance}
    sparsifier.Compute();
    \textcolor{comment}{// Perform Sparsification }
    \textcolor{keyword}{auto} sparseGraph1 = sparsifier.GetSparsifiedGraph();
    std::cout<<\textcolor{stringliteral}{"Sparised Edge Count (ApproxER): "} << sparseGraph1->GetEdgeCount() << std::endl;

    \textcolor{comment}{// Set ER Policy to Exact Effective Resistance instead of Approximation}
    \textcolor{comment}{// This is expected to perform much slower.}
    sparsifier.SetERPolicy(gSparse::SpectralSparsifier::EXACT\_ER);
    sparsifier.Compute();
    \textcolor{keyword}{auto} sparseGraph2 = sparsifier.GetSparsifiedGraph();
    std::cout<<\textcolor{stringliteral}{"Sparised Edge Count (ApproxER): "} << sparseGraph2->GetEdgeCount() << std::endl;
    \textcolor{keywordflow}{return} EXIT\_SUCCESS;
\}
\end{DoxyCode}


\subsection*{Examples-\/5}

This example perform Spectral Sparsification, and use Spectra to calcualte Eigenvalue of Graph Laplacians


\begin{DoxyCode}
\textcolor{preprocessor}{#include <gSparse/gSparse.hpp>}
\textcolor{preprocessor}{#include <iostream>}
\textcolor{preprocessor}{#include <Eigen/Core>}
\textcolor{preprocessor}{#include <Eigen/SparseCore>}
\textcolor{preprocessor}{#include <Spectra/GenEigsSolver.h>}
\textcolor{preprocessor}{#include <Spectra/MatOp/SparseGenMatProd.h>}

Eigen::VectorXcd GetTopEigenValue(\textcolor{keyword}{const} gSparse::SparsePrecisionMatrix & M)
\{
    std::cout << \textcolor{stringliteral}{"Calculating Eigen Value"}<<std::endl;
    \textcolor{keyword}{using namespace }Spectra;
    SparseGenMatProd<double> op(M);

    \textcolor{comment}{// Construct eigen solver object, requesting the largest three eigenvalues}
    GenEigsSolver< double, LARGEST\_MAGN, SparseGenMatProd<double> > eigs(&op, 3, 6);

    \textcolor{comment}{// Initialize and compute}
    eigs.init();
    \textcolor{keywordtype}{int} nconv = eigs.compute();

    \textcolor{comment}{// Retrieve results}
    Eigen::VectorXcd evalues;
    \textcolor{keywordflow}{if}(eigs.info() == SUCCESSFUL)
        evalues = eigs.eigenvalues();

    std::cout << \textcolor{stringliteral}{"Eigenvalues found:\(\backslash\)n"} << evalues << std::endl;
    \textcolor{keywordflow}{return} evalues;
\}
\textcolor{keywordtype}{int} main()
\{
    \textcolor{comment}{// Generate 100x100 Complete Graph}
    \textcolor{keyword}{auto} graph = gSparse::Builder::buildUnitCompleteGraph(100);
    \textcolor{comment}{// Creating Sparsifier Object}
    \mbox{\hyperlink{classg_sparse_1_1_spectral_sparsifier_1_1_e_r_sampling}{gSparse::SpectralSparsifier::ERSampling}} sparsifier(graph);
    \textcolor{comment}{// Set Hyper-parameters}
    \textcolor{comment}{// Approximate the Effective Weight Resistance (Faster)}
    \textcolor{comment}{// C = 100 and Epsilon = 0.5}
    sparsifier.SetERPolicy(gSparse::SpectralSparsifier::APPROXIMATE\_ER);
    sparsifier.SetC(100.0);
    sparsifier.SetEpsilon(0.5);
    \textcolor{comment}{// Compute Effective Weight Resistance using ApproxER}
    sparsifier.Compute();
    \textcolor{comment}{// Get a sparsified graph}
    \textcolor{keyword}{auto} sparseGraph1 = sparsifier.GetSparsifiedGraph();
    \textcolor{comment}{// Set to EXACT ER}
    sparsifier.SetERPolicy(gSparse::SpectralSparsifier::EXACT\_ER);
    \textcolor{comment}{// Re-calcuate ER using ExactER}
    sparsifier.Compute();
    \textcolor{comment}{// Get a sparsified graph}
    \textcolor{keyword}{auto} sparseGraph2 = sparsifier.GetSparsifiedGraph();

    \textcolor{comment}{// Use Spectra to calculate top Eigen values. }
    std::cout<<\textcolor{stringliteral}{"Original Eigen Value "} <<std::endl;
    GetTopEigenValue(graph->GetLaplacianMatrix());
    std::cout<<\textcolor{stringliteral}{"---------------------------"}<<std::endl;
    std::cout<<\textcolor{stringliteral}{"Top Eigen Value for ApproxER "}<<std::endl;
    GetTopEigenValue(sparseGraph1->GetLaplacianMatrix());
    std::cout<<\textcolor{stringliteral}{"Top Eigen Value for ApproxER - Original"}<<std::endl;
    GetTopEigenValue(sparseGraph1->GetLaplacianMatrix() - graph->GetLaplacianMatrix());
    std::cout<<\textcolor{stringliteral}{"---------------------------"}<<std::endl;
    std::cout<<\textcolor{stringliteral}{"Top Eigen Value for ExactER "}<<std::endl;
    GetTopEigenValue(sparseGraph2->GetLaplacianMatrix());
    std::cout<<\textcolor{stringliteral}{"Top Eigen Value for ApproxER - ExactER"}<<std::endl;
    GetTopEigenValue(sparseGraph1->GetLaplacianMatrix() - graph->GetLaplacianMatrix());
    \textcolor{keywordflow}{return} 0;
\}
\end{DoxyCode}
 