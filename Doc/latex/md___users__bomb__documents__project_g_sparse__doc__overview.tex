{\bfseries g\+Sparse\+:} C++ Library for Graph Sparsification

{\bfseries g\+Sparse} is a C++ library for Graph Sparsification. It is built on top of \href{http://eigen.tuxfamily.org}{\tt Eigen}, an open source linear algebra library.

\subsection*{Sparsification Algorithm(s)}

{\bfseries g\+Sparse} currently supports the following algorithm(s).
\begin{DoxyItemize}
\item \href{https://arxiv.org/abs/0803.0929}{\tt Spectral Sparsification by Effective Resistance}
\end{DoxyItemize}

\subsection*{Header-\/only Library}

{\bfseries g\+Sparse} is implemented as a header-\/only C++ library, whose only dependency, {\bfseries Eigen}, is also header-\/only.

\subsection*{Relation to Fast\+Effective\+Resistance}

\href{http://www.cs.cmu.edu/~jkoutis/SpectralAlgorithms.htm}{\tt Fast\+ER} is a software written in M\+A\+T\+L\+AB for calculating Graph Effective Resistance. {\bfseries g\+Sparse} leverages the algorithm of Fast\+ER for Spectral Sparsification in C++. However, {\bfseries g\+Sparse} is not a direct port of Fast\+ER because it provides a different interface and it does not depend on C\+MG Solver.

\subsection*{Examples}

Below is an example that demonstrates the g\+Sparse to sparsify a graph from C\+SV files


\begin{DoxyCode}
\textcolor{preprocessor}{#include <gSparse/gSparse.hpp>}
\textcolor{preprocessor}{#include <iostream>}

\textcolor{keywordtype}{int} main()
\{
    \textcolor{comment}{// Create a CSV Reader and specify the file location}
    gSparse::GraphReader csvReader = std::make\_shared<gSparse::GraphCSVReader>(\textcolor{stringliteral}{"edgeList.csv"}, \textcolor{stringliteral}{"
      weightList.csv"},\textcolor{stringliteral}{","});
    \textcolor{comment}{// Create an Undirected Graph. gSparse::Graph object is a shared\_ptr.}
    gSparse::Graph graph = std::make\_shared<gSparse::UndirectedGraph>(csvReader);

    \textcolor{comment}{// Display original edge count}
    std::cout<<\textcolor{stringliteral}{"Original Edge Count: "} << graph->GetEdgeCount() << std::endl;

    \textcolor{comment}{// Creating Sparsifier Object}
    \mbox{\hyperlink{classg_sparse_1_1_spectral_sparsifier_1_1_e_r_sampling}{gSparse::SpectralSparsifier::ERSampling}} sparsifier(graph);
    \textcolor{comment}{// Set Hyper-parameters}
    \textcolor{comment}{// Approximate the Effective Weight Resistance (Faster)}
    \textcolor{comment}{// C = 4 and Epsilon = 0.5}
    sparsifier.SetERPolicy(gSparse::SpectralSparsifier::APPROXIMATE\_ER);
    sparsifier.SetC(4.0);
    sparsifier.SetEpsilon(0.5);
    \textcolor{comment}{// Compute Effective Weight Resistance}
    sparsifier.Compute();
    \textcolor{comment}{// Perform Sparsification }
    \textcolor{keyword}{auto} sparseGraph1 = sparsifier.GetSparsifiedGraph();
    std::cout<<\textcolor{stringliteral}{"Sparised Edge Count (ApproxER): "} << sparseGraph1->GetEdgeCount() << std::endl;
    \textcolor{keywordflow}{return} EXIT\_SUCCESS;
\}
\end{DoxyCode}


\subsection*{Documentation}

The \href{https://codedocs.xyz/As-12/gSparse/}{\tt A\+PI reference} page contains the documentation of {\bfseries g\+Sparse} generated by \href{http://www.doxygen.org/}{\tt Doxygen}, including all the background knowledge, example code and class A\+P\+Is.

\subsection*{License}

{\bfseries g\+Sparse} is an open source project licensed under M\+IT License. 